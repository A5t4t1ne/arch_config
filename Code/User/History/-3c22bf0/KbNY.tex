\documentclass[12pt]{scrartcl}
\usepackage[ngerman]{babel}


\usepackage{amsmath, amssymb}

\usepackage{array}  % for the tables

\usepackage{nameref}  % for referencing with name

\usepackage{hyperref}  % for hyperlinks

\usepackage{mathrsfs}

\usepackage{graphicx}  % for the images

\usepackage{xcolor, colortbl}

\usepackage{gensymb} % for \degree

\usepackage{pgfplots}

\usepackage{tabto}

\newcolumntype{P}[1]{>{\centering\arraybackslash}p{#1}}

\usetikzlibrary{arrows}

% \usepgfplotslibrary{external}

% \tikzexternalize

\definecolor{Gray}{gray}{0.85}

% \setlength{\parindent}{0cm}

% hyperlinks
\hypersetup{
    colorlinks,
    citecolor=black,
    filecolor=black,
    linkcolor=black,
    urlcolor=black
}

\bibliographystyle{IEEetran}




\author{David Jäggli}

\title{Formelsammlung Krypto}



% ---------- Begin Main Document ----------- %



\begin{document}

\maketitle

\tableofcontents

\newpage
\section{Allg}


\section{Terminologie}


\renewcommand{\arraystretch}{1.5}
\begin{center}
    \begin{tabular}{ | m{12em} | m{25em} | }
        \hline
        \textbf{Kryptographie}         & Entwerfen von Krypto-Algorithmen                       \\
        \hline
        \textbf{Kryptoanalyse}         & Brechen von Krypto-Algorithmen                         \\
        \hline
        \textbf{Perfekte Sicherheit}   & Unendlich viele Ressourcen sind equivalent zu raten    \\
        \hline
        \textbf{Unkeyed Kryptographie} & Hashfunktionen                                         \\
        \hline
        \textbf{Symmetrische Krypt.}   & Beide den gleichen Schlüssel - $\mathcal{O}(n²)$       \\
        \hline
        \textbf{Asymmetrische Krypt.}  & Öffentlicher und privater Schlüssel - $\mathcal{O}(n)$ \\
        \hline
    \end{tabular}
\end{center}


\section{Symmetrische Kryptographie}

\section{Asymmetrische Kryptographie}

\newpage
\section{Blinde Signaturen}

Generelle Beschreibung: Anna weis nicht WAS sie unterschreibt, wenn sie das Dokument später sieht,
weis sie aber DASS sie es unterschrieben hat.\\
Nutzen:
\begin{itemize}
    \item Unverfällschbarkeit
    \item Anonymität
    \item Unlinkbarkeit
\end{itemize}

\vspace{0.5cm}
\noindent
Ablauf:
\label{sec:dinimam}
\begin{enumerate}
    \item Kunde zieht geld ab
    \item Bank signiert den Betrag
    \item Kunde bezahlt im Shop
    \item Shop schickt die Unterschrift an die Bank
    \item Bank prüft Unterschrift
    \item Bank validiert Unterschrift und zieht Geld ab
\end{enumerate}

\vspace{0.5cm}
Beispiel: Siehe s.100 Folien 09


\section{Einführung in die Public-Key Infrastruktur (PKI)}

\subsection{Verschlüsseln und Signieren (repetition)}

% TODO: verlinken zu initialem Kapitel

\subsubsection{Verschlüsseln}

\subsubsection{Signieren}

\textbf{Ablauf signieren:}

\begin{enumerate}
    \item Dokument von Alice ist Ausgangswert
    \item Hash berechnen $\rightarrow$ Hashwert
    \item chiffrieren (mit private key und Hash) $\rightarrow$ Signatur
    \item Dokument \& Signatur + Zertifikat $\rightarrow$ signiertes Dokument
\end{enumerate}

Warum Zertifikat? $\rightarrow$ um sicherzustellen, dass der öffentliche Schlüssel auch wirklich
von Alice ist.

\newpage
\textbf{Ablauf Signatur prüfen:}
\begin{enumerate}
    \item Dokument von Alice ist Ausgangswert
    \item Signatur mit öffentlichem Schlüssel entschlüsseln $\rightarrow$ Hashwert
    \item Hashwert von Dokument berechnen
    \item Hashes vergleichen
    \item Zertifikat Überprüfen
\end{enumerate}


\subsection{Zertifikate}

\subsubsection{Herstellung eines Zertifikats}

Zertifikatsinhalt:

\begin{enumerate}
    \item \begin{itemize}
              \item Version
              \item Serial Number
              \item Subject
              \item Public Key
          \end{itemize}
    \item yup
\end{enumerate}




% ---------- End Main Document ----------- %


% image example
% \begin{figure}[ht]
%     \centering
%     \includegraphics[width=\textwidth]{img/Anwendungsbeispiel_signieren.png}
%     \caption{Anwendungsbeispiel Signieren}
%     \label{fig:your_label}
% \end{figure}



% \begin{align*}
%     587 &= 1 \cdot 392 + 195 \\
% \end{align*}

% Matrix example
% $
% \begin{bmatrix}
%     a_{1,1} & a_{1,2} & a_{1,n}\\
%     a_{2,1} & a_{2,2} & a_{2,n}\\
%     a_{m,1} & a_{m,2} & a_{m,n}
% \end{bmatrix}$\\

% tabular example 3 columns
% \renewcommand{\arraystretch}{1.5}
% \begin{center}
%     \begin{tabular}{ | m{12em} | m{12em} | m{12em} | }
%         \hline
%         1 & 2 & 3\\
%         \hline
%         1 & 2 & 3\\ 
%         \hline
%         1 & 2 & 3\\
%         \hline
%     \end{tabular}
% \end{center}


% tabular example 2 columns
% \renewcommand{\arraystretch}{1.5}
% \begin{center}
%     \begin{tabular}{ | m{17em} | m{17em} | }
%         \hline
%         1 & 2\\
%         \hline
%         1 & 2\\
%         \hline
%         1 & 2\\
%         \hline
%     \end{tabular}
% \end{center}


% \begin{tikzpicture}[line cap=round,line join=round,>=triangle 45,x=0.5cm,y=0.25cm]
%     \begin{axis}[
%     x=0.75cm,y=0.5cm, % size of the grid
%     axis lines=middle,
%     ymajorgrids=true,
%     xmajorgrids=true,
%     xmin=-10,
%     xmax=10,
%     ymin=-10,
%     ymax=10,
%     xtick={-11,-10,...,10},
%     ytick={-10,-9,...,9},]
%     \draw[line width=2pt,color=blue] (-10,-5) -- (-2,-1);
%     \begin{scriptsize}
%         \draw[color=blue] (-9.866,-4.728) node {$g$};
%         \draw[color=blue] (-1.906,7.172) node {$f$};
%         \draw[color=blue] (3.134,5.232) node {$h$};
%     \end{scriptsize}
% \end{axis}
% \end{tikzpicture}




% \bibliography{quantum_ready}

\end{document}